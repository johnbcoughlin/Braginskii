\documentclass{article}

\title{IMEX schemes for the electrostatic magnetized Vlasov equation with Dougherty-Fokker-Planck collisions}

\begin{document}
\maketitle

The governing equation is
\begin{align}
    \label{eqn:vlasov}

    \partial_t f_\alpha + v \cdot \nabla_x f_\alpha + (E + v \times B) \cdot \nabla_v f_\alpha = Q(f_\alpha),
\end{align}
where 
\begin{align}
    Q(f_\alpha) = \sum_\beta Q(f_\alpha, f_\beta), \quad Q(f_\alpha, f_\beta) = \nu_{\alpha\beta} \nabla_v \cdot (T_{\alpha\beta} \nabla_v f_\alpha + (v - u_{\alpha\beta}) f_\alpha).
\end{align}
We suppose that the magnetic field $B$ is constant and relatively strong (low-beta regime).
We are interested in regimes where both the magnetization and collisionality is moderate to strong.
Particularly for physical mass ratios, the timestep restriction in such regimes is dominated by the magnetic and collisional terms.
This motivates the development of IMEX schemes for \eqref{eqn:vlasov}, wherein the magnetic component of the Lorentz force term and 
the collisional term are treated implicitly.



\end{document}
